\documentclass{amsart}


\usepackage[utf8]{inputenc}
\usepackage{color}
\usepackage{hyperref}
\usepackage{tikz-cd}
\usepackage{mathtools}
\usepackage{csquotes}
\usepackage{tikz}
\usepackage{enumerate}
\usetikzlibrary{arrows}

\newcommand{\ds}[1]{\ensuremath{ \displaystyle{#1} }}

\title{Topologies on Schemes}
\author{Jacob Hegna}
\date{\today}

\newcommand{\Zz}{\mathbb{Z}}
\newcommand{\Gg}{\mathbb{G}}
\newcommand{\Hh}{\mathbb{H}}
\newcommand{\Top}{\mathbf{Top}}
\newcommand{\Sch}{\mathbf{Sch}}
\newcommand{\Aff}{\mathbf{Aff}}
\newcommand{\Ring}{\mathbf{Ring}}
\newcommand{\Set}{\mathbf{Set}}
\newcommand{\Sh}{\mathbf{Sh}}
\newcommand{\Ab}{\mathbf{Ab}}
\newcommand{\PAb}{\mathbf{PAb}}

\DeclareMathOperator{\Zar}{Zar}
\DeclareMathOperator{\Fun}{Fun}
\DeclareMathOperator{\Cov}{Cov}
\DeclareMathOperator{\Cl}{Cl}
\DeclareMathOperator{\St}{St}
\DeclareMathOperator{\link}{link}
\DeclareMathOperator{\Sub}{Sub}
\DeclareMathOperator{\Spec}{Spec}
\DeclareMathOperator{\Ann}{Ann}
\DeclareMathOperator{\Mod}{Mod}
\DeclareMathOperator{\mmod}{mod}
\DeclareMathOperator{\coker}{coker}
\DeclareMathOperator{\Comm}{Comm}
\DeclareMathOperator{\conv}{conv}
\DeclareMathOperator{\diag}{diag}
\DeclareMathOperator{\Ext}{Ext}
\DeclareMathOperator{\Tor}{Tor}
\DeclareMathOperator{\Gr}{Gr}
\DeclareMathOperator{\Hom}{Hom}
\DeclareMathOperator{\im}{im}

\newtheorem{theorem}{Theorem}[section]
\newtheorem{proposition}[theorem]{Proposition}
\newtheorem{lemma}[theorem]{Lemma}
\newtheorem{corollary}[theorem]{Corollary}
\theoremstyle{definition}
\newtheorem{definition}[theorem]{Definition}
\newtheorem{example}[theorem]{Example}
\newtheorem{remark}[theorem]{Remark}
\theoremstyle{remark}

\newcommand{\abs}[1] {
  \left| #1 \right|}

\newcommand{\norm}[1] {
  \left| \abs{#1} \right|}

\newcommand\restr[2]{{% we make the whole thing an ordinary symbol
  \left.\kern-\nulldelimiterspace % automatically resize the bar with \right
  #1 % the function
  \vphantom{|} % pretend it's a little taller at normal size
  \right|_{#2} % this is the delimiter
  }}

\begin{document}

\maketitle

\begin{abstract}
    This short note aims to be a readable exposition of the basics of
    alternative (Grothendieck) (pre)topologies on the category of schemes.
    Proofs are provided where they provide intuition. Definitions are provided
    for all material one would not likely encounter in an introductory
    commutative algebra and geometry course, which makes the intended audience
    graduate students who have finished these courses.
\end{abstract}

\tableofcontents

\section{Introduction}

Often, when dealing with a geometric object $X$, one studies its properties by
endowing it with some topology and computing invariants
$H^\ast_{\text{singular}}$. This is useful, for instance, in distinguishing
various affine spaces. Endowed with the standard metric topology, we have that
\[
    H_{\text{singular}}^i(\mathbb{C}^n \setminus \{p\}, \Zz) =
    \begin{cases}
        \Zz, \quad i = 2n \\
        0, \quad \text{else}
    \end{cases},
\]
for any $p \in \mathbb{C}^n$. This implies that there is no homeomorphism from
$\mathbb{C}^n$ to $\mathbb{C}^m$ if $m \neq n$. We would like to use similar
invariants to study objects in other categories, namely $\Sch$. Unfortunately,
the singular invariants are not well-behaved in the Zariski topology. Namely,
the singular invariants of a scheme $X$ factor through the forgetful functor to
$\mathbf{Top}$, but this implies $H^\ast_\text{singular}(\mathbb{A}^1) =
H^\ast_\text{singular}(\mathbb{P}^1)$ as the Zariski topology of the projective
and affine line are each simply the cofinite topology (over any closed field).
Indeed, the situation is quite bad for studying any purely topological
invariants of a scheme, which is demonstrated by the following proposition.

\begin{proposition}
    Let $k$ be an algebraically closed field of cardinality greater than or
    equal to $\abs{\mathbb{R}}$ and let $X$ be an irreducible variety over $k$.
    Then, $X$ is topologically contractible.
\end{proposition}

Thus, it is clear that the elementary tools afforded to us by algebraic topology
will not suffice. Instead, we turn to sheaf cohomology: given $\mathcal{F} \in
\Ab(X)$, it is a result of Grothendieck that there exists an injective
resolution of $\mathcal{F}$. We apply global sections to this resolution, and
take its cohomology. These groups, $H^i(X, \mathcal{F})$ are useful invariants
of a scheme. For instance, the vanishing of these groups in every dimension for
$\mathcal{F}$ quasicoherent is equivalent to $X = \Spec(R)$.  However, these
groups are not quite sufficient for our needs. For instance, the cohomology of
projective space with coefficients in any constant sheaf vanishes, which is not
analogous to the singular case:
$H^{2i}_{\text{singular}}(\mathbb{P}^n(\mathbb{C}), \Zz_p) = \Zz_p$ if and only
if $i = n$. This expository paper aims to develop a cohomology theory which
captures this behavior.

\section{Topologies on categories}

The end goal of this paper is to develop a cohomology theory that more closely
models cohomology in the analytic category. Seeing as (Zariski) sheaf cohomology
satisfies many desireable properties, it is natural to consider ways to
generalize this that preserve the nice theoretical properties of Zariski
cohomology while also yielding the results we want, particularly on projective
space.

\subsection{Pretopologies}

\begin{definition}
    A \textit{Grothendieck pretopology} on a category $\mathcal{C}$ equipped
    with fiber products is a set of distinguished sets of morphisms $\{X_\alpha
    \to X\}_{\alpha \in \Lambda}$ called \textit{coverings} such that:
    \begin{enumerate}
        \item Every isomorphism is a covering.
        \item If $\{X_\alpha \to X\}_{\alpha \in \Lambda}$ is a covering and,
            for each $\alpha$, $\{Y_{\alpha \beta} \to X_\alpha\}_{\beta \in
                \Gamma}$ is a covering, then $\{Y_{\alpha \beta} \to
            X\}_{(\alpha, \beta) \in \Lambda \times \Gamma}$ is a covering.
        \item If $Y \to X$ is a morphism
            and $\{X_\alpha \to X\}_{\alpha \in
                \Lambda}$ is a covering, then $\{X_\alpha \times_{X} Y \to
            Y\}_{\alpha \in \Lambda}$ is a covering.
    \end{enumerate}
    If $\{X_\alpha \to X\}_{\alpha \in \Lambda}$ is a covering, we say
    $\{X_\alpha\}$ \textit{covers} $X$.
\end{definition}

Observe that we require that each collection is indeed a set---for now, note
that this is nontrivial. Given a category $\mathcal{C}$ with fibre products and
where the collection of objects is a set and the collection $\Hom(X, Y)$ is a
set, we automatically have the following:

\begin{proposition}
    Given the assumptions above, the following two desciptions of coverings
    define Grothendieck pretopologies on $\mathcal{C}$:
    \begin{enumerate}
        \item $\{\phi : X \to Y\}$ where $\phi$ is an isomorphism.

        \item $\{\psi : X \to Y\}$ where $\psi \in \Hom(X, Y)$.
    \end{enumerate}
    We say the first is the \textit{discrete} pretopology and the second is the
    \textit{indiscrete} pretopology.
\end{proposition}

\begin{proof}
    Each automatically satisfies the first axiom of a Grothendieck pretopology.
    The indiscrete pretopology trivially satisfies the second and third axioms
    automatircally. Isomorphisms compose to isomorphisms, so the discrete
    pretopology satisfies the second axiom. For the third axiom, consider that
    if $X' \to X$ is an isomorphism and $Y \to X$ a morphism, then $Y \times_X
    X' = Y \times_X X = Y$, so $\{Y \times_X X' \to Y\}$ is a covering.
\end{proof}

\subsection{Sites}

\begin{definition}
    A \textit{site} is a pair $(\mathcal{C}, \Cov(\mathcal{C}))$ where
    $\mathcal{C}$ is a category and $\Cov(\mathcal{C})$ is a Grothendieck
    pretopology on $\mathcal{C}$.
\end{definition}

For any topological space $X$, we define the (little) Zariski site on $X$ to be the
category whose objects are open subsets of $X$ and morphisms are inclusions of
open sets, equipped with the indiscrete pretopology. It is denoted
$X_{\text{Zar}}$. We define the big Zariski site of $X$ to be the comma category
$\Top/X$ equipped with the pretopology consisting of families of open immersions
sharing a common codomain, whose union of images equal the codomain. Recall an
\textit{open immersion} of topological spaces is a morphism which is open and a
homeomorphism on its image.

Unfortunately, the Zariski sites on a scheme $X$ are not recovered by the
forgetful functor $\Sch \to \Top$. However, the sites are defined analogously
as the previous case. The little Zariski site of $X$, denoted $X_{\text{\'et}}$
is the category whose objects are schemes $Y$ over $X$ whose structure morphism
is an open immersion. The big Zariski site, denoted $(\Sch/X)_{\text{\'et}}$ is
defined as the category $\Sch/X$ with the Zariski topology.  In each case, the
coverings are families of open immersions whose image set-theoretically cover
$X$.

In general, given a ``nice'' class of morphisms $\tau$ and a scheme $X$, we
define the \textit{little $\tau$-site} to be the category whose objects are
schemes over $X$ whose structure morphism is $\tau$, and coverings are families
of $\tau$-morphisms whose image set theoretically cover $X$. The \textit{big
    $\tau$-site} is defined as the category $\Sch/X$ with the same notion of
covering as in the little site. We denote the little site by $X_\tau$ and the
big site by $(\Sch/X)_\tau$.

Generally, the big site will reveal more information about the underlying scheme
$X$, but will pose more technical problems. The little site will behave more
closely to objects in classical geometry.

\begin{remark}
    There is an obvious set-theoretic problem here if we work with the naive
    definitions of $\Top$ and $\Sch$ (the collection of objects is indeed a
    proper class), however this will be resolved in a later section.
\end{remark}

\begin{remark}
    In older texts (namely EGA and SGA), the big and small sites are referred to
    as the \textit{gros} and \textit{petit} site, respectively.
\end{remark}

\begin{example}
    Let $G$ be a group and denote by $\mathcal{T}_G$ the category of $G$-sets.
    Let coverings in this category be families of $G$-equivariant maps with
    shared image such that the family set-theoretically covers this image. In
    other words, a family of $G$-equivariant maps $\{U_i \to U\}$ is a covering
    iff $\bigcup U_i = U$.
    
    This is a well-known example of a site, and this particular telling of the
    story follows that of
    \cite[\href{https://stacks.math.columbia.edu/tag/03NP}{Tag
        03NP}]{stacks-project}.
\end{example}

\section{Sheaves}

For this section, we assume all categories enjoy all finite fibre products. This
is not strictly necessary, but makes the exposition straightfoward. A
\textit{presheaf} on a category $\mathcal{C}$ is an object of the functor
category $\Fun(\mathcal{C}^{\text{op}}, \Set)$. The category of presheaves on
$\mathcal{C}$ should be seen as the canonical co-completion of $\mathcal{C}$.

\begin{definition}
    Let $\mathcal{C}$ be a site and $\mathcal{F}$ a presheaf on $\mathcal{C}$.
    We say that $\mathcal{F}$ is a \textit{sheaf} if, for every covering $\{U_i
    \to U\}_{i \in I}$, the following diagram
    \[
        \begin{tikzcd}
            \mathcal{F}(U) \ar[r] &
            \prod_{i \in I} \mathcal{F}(U_i) \ar[r, shift left=0.75ex] \ar[r,
            shift right=0.75ex] &
            \prod_{i, j \in I} \mathcal{F}(U_i \times_U U_j),
        \end{tikzcd}
    \]
    is an equalizer in $\Set$. We refer to the category of sheaves on a site by
    $\Sh(\mathcal{C}$. Moreover, any category $\mathcal{T}$ equipped with an
    equivalence to $\Sh(\mathcal{D})$ for some site $\mathcal{D}$ is defined to
    be a \textit{(Grothendieck) topos}.
\end{definition}

We now pivot to focus our attention on the category $\Sch$. Let $\mathcal{F}$ be
a sheaf on $X$ a scheme. We say that $\mathcal{F}$ is \textit{quasicoherent} if,
for each $\Spec(A) \subset X$, $\restr{\mathcal{F}}{\Spec(A)} \cong
\widetilde{M}$ for an $A$-module $M$. It is \textit{coherent} if furthermore
$M$ is finitely presented.

The following proposition requires notions developed later, but it is best
placed here for reference.

\begin{proposition}
    For any quasicoherent sheaf $\mathcal{F}$ on $S$, the functor
    \begin{align*}
        \mathcal{F}^a : \Sch/S &\to \Ab \\
        (f : T \to S) &\mapsto \Gamma(T, f^{\ast}\mathcal{F})
    \end{align*}
    is an $\mathcal{O}$-module which satisfies the sheaf condition in the fpqc
    topology.
\end{proposition}

\begin{proof}
    See \cite[\href{https://stacks.math.columbia.edu/tag/03OG}{Tag
        03OG}]{stacks-project}.
\end{proof}

\section{Morphisms in $\Sch$}

\subsection{Finiteness properties}

These definitions are standard in the literature. We collect them here for the
reader's convenience.

\begin{definition}
    Let $f : X \to Y$ be a morphism of schemes.

    \begin{enumerate}[(i)]
        \item $f$ is \textit{locally of finite type} if there exists an open
            cover $\{V_i\}$ of $Y$ such that $V_i = \Spec(B_i)$ and for each
            $i$, $f^{-1}(V_i)$ can be covered by open sets $U_j = \Spec(A_j)$ so
            that $A_j$ is a finitely generated $B_i$-algebra.

        \item $f$ is \textit{of finite type} if there exists an open cover
            $\{V_i\}$ of $Y$ such that $V_i = \Spec(B_i)$ and for each $i$,
            $f^{-1}(V_i)$ can be covered by \textbf{finitely many} open sets
            $U_j = \Spec(A_j)$ so that $A_j$ is a finite generated
            $B_i$-algebra.

        \item $f$ is \textit{finite} if there exists an open cover $\{V_i\}$ of
            $Y$ such that $V_i = \Spec(B_i)$ and $f^{-1}(V_i) = \Spec(A_i)$ for
            $A_i$ a finitely generated $B_i$-algebra.
    \end{enumerate}
\end{definition}

\subsection{Smooth morphisms}

\begin{definition}
    A morphism $f : X \to Y$ is \textit{smooth of relative dimension $n$} if it
    satifies either of the following equivalent conditions:
    \begin{enumerate}[(i)]
        \item \textbf{(Local Jacobian condition)}
            There exist open covers $\{V_i\}$ of $Y$ and $\{U_i\}$ of $X$ such
            that, for every $i$, there is a commutative diagram
            \[
                \begin{tikzcd}
                    U_i \ar[d, "\restr{f}{U_i}"] \ar[r, "\sim"] &
                    W \ar[d, "\restr{\rho}{W}"] \\
                    V_i \ar[r, "\sim"] &
                    \Spec(B)
                \end{tikzcd}
            \]
            where there exists a morphism $\Spec(B[x_1, \dots, x_{n+r}]/(f_1, \dots
            f_r)) \to \Spec(B)$ induced by the inclusion of the coefficient ring, and
            $W$ is an open subscheme of $\Spec(B[x_1, \dots, x_{n+r}]/(f_1, \dots,
            f_r))$. Moreover, we require the morphism $\rho$ to satisfy the
            \textit{Jacobian condition}, the determinant of the matrix of
            partials of $f_i$ with respect to the first $r$ variables of $B[x_1,
            \dots, x_{n+r}$ is everywhere invertible, that is,
            \[
                \det \left( \frac{\partial f_i}{\partial x_j} \right)
            \]
            is a nowhere zero function on $W$.

        \item \textbf{(Algebraic condition)} $f$ is locally of finite
            presentation, $f$ is flat of relative dimension $n$, and
            $\Omega_{X/Y} = 0$.

    \end{enumerate}

    We say a morphism is \textit{smooth} if it is smooth of some arbitrary
    dimension $n$.
\end{definition}

\begin{theorem}
    The above definitions are indeed equivalent.
\end{theorem}

\begin{proof}
    Omitted. See \cite{vakil}.
\end{proof}

\begin{remark}
    By either definition, smoothness is local on both the source and target. A
    consequence of this is that the locus where a morphism is smooth is open!
    Moreover, a smooth morphism is necessarily locally of finite presentation.
\end{remark}

\begin{proposition}
    Let $f : X \to Y$ be a smooth morphism of relative dimension $n$ where $X$
    and $Y$ are schemes over $S$. Then, the base-change $f \times Z : X \times Z
    \to Y \times Z$ is smooth of relative dimension $n$.
\end{proposition}

\begin{proof}
    We use the algebraic condition for smoothness. Each of the criteria (locally
    of finite presentation, flat of dimension $n$, and vanishing of the sheaf of
    differentials) is preserved by base change by arbitrary schemes.
\end{proof}

\begin{proposition}
    Let $f : X \to Y$ be smooth of relative dimension $n$ and $f : Y \to Z$
    smooth of relative dimension $m$. Then, $g \circ f$ is smooth of relative
    dimension $n + m$.
\end{proposition}

\begin{proof}
    Omitted. See \cite{vakil}.
    \iffalse
    The question is local, so we write $X = \Spec(A)$, $Y = \Spec(B)$, and $Z =
    \Spec(C)$. Consider the following diagram
    \[
        \begin{tikzcd}
            \Spec(A) \ar[r, "\sim"] \ar[d, "f"] &
            W_B \ar[d, "\restr{\rho_B}{W_B}"] \\
            \Spec(B) \ar[r, "\text{id}"] \ar[d, "\text{id}"] &
            \Spec(B) \ar[d, "\rotatebox{90}{$\sim$}"] \\
            \Spec(B) \ar[r, "\sim"] \ar[d, "g"] &
            W_C \ar[d, "\restr{\rho_C}{W_C}"] \\
            \Spec(C) \ar[r, "\text{id}"] &
            \Spec(C)
        \end{tikzcd}
    \]
    The upper and lower squares commute by hypothesis. The middle row commutes
    by construction. The outer rectangle is the desired commutative square in
    the definition of smoothness. To show the smoothness of $f \circ g$ is of
    relative dimension $n + m$, we observe that the map $\rho_B \circ \rho_C$ is
    induced by successive inclusions of coordinate rings $C \xhookrightarrow{}
    C[x_1, \dots, x_{m + s}]/(f_1, \dots, f_s)$ and $B \xhookrightarrow{} B[y_1,
    \dots, y_{n + r}]/(g_1, \dots, g_r)$.
    \fi
\end{proof}

\subsection{Flat morphisms}

\begin{displayquote}[David Mumford, \textit{The Red Book}]
    ``The concept of flatness is a riddle that comes out of algebra, but which
    technically is the answer to many prayers''
\end{displayquote}

\begin{definition}
    A $R$-module $M$ is \textit{flat} if any of the following equivalent
    conditions are satisfied:
    \begin{enumerate}[(i)]
        \item For $0 \to A \to B \to C \to 0$ exact, $0 \to A \otimes M \to B
            \otimes M \to C \otimes M \to 0$ is exact.
        \item The functor $M \otimes {-}$ takes monomorphisms to monomorphisms.
        \item $\Tor^R_1(M, N) = 0$ for all $R$-modules $N$.
        \item $\Tor^R_i(M, N) = 0$ for all $i > 0$ and $R$-modules $N$.
    \end{enumerate}
    These conditions say the functor $M \otimes {-}$ is \textit{exact}. We say
    that $M$ is \textit{faithfully flat} if the converse to (i) holds.
\end{definition}

\begin{definition}
    A morphism of rings $f : A \to B$ is \textit{flat} (resp. \textit{faithfully
        flat}) if $B$ enjoys the structure of a flat (resp. faithfully flat)
    $A$-module under $f$.
\end{definition}

\begin{definition}
    Let $f : X \to Y$ be a morphism of schemes and $\mathcal{F}$ quasicoherent
    on $X$. We say $\mathcal{F}$ is \textit{flat} over $Y$ at $x \in X$ if
    $\mathcal{F}_x$ is flat as a $\mathcal{O}_{Y, f(x)}$ module (with the
    structure morphism being given by $\mathcal{O}_{Y, f(x)} \to \mathcal{O}_{X,
        x}$). The map $f$ is \textit{flat} at $x \in X$ if $\mathcal{O}_X$ is
    flat over $Y$ at $x$. $f$ is \textit{flat} if it is flat at every point of
    $X$. $f$ is \textit{faithfully flat} if it is flat and surjective.
\end{definition}

We now list helpful properties of flatness taken from \cite{MorphismNotes},
without proof.

\begin{enumerate}
    \item The composition of two flat morphisms is flat.
    \item The fibre product of two flat (resp. faithfully flat) morphisms remain
        flat (resp. faithfully flat).
    \item Flatness and faithful flatness are preserved by base change.
    \item For $f$ finitely presented, flatness is an \textit{open} condition
        (the set $\{x \in X \mid f \text{ is flat at } x \}$ is open, although
        possibly empty).
    \item Open immersions are flat (they induced isomorphisms on stalks).
    \item The pullback functor $\mathit{QCoh}(Y) \to \mathit{QCoh}(X)$ is
        exact if induced by a flat morphism.
    \item Flatness and locally of finite presentation imply open.
    \item \textbf{(Generic flatness)} If $f : X \to Y$ is of finite type and $Y$
        is integral, then there is a nonempty open subset $U \subset Y$ such
        that the restrict of $f$ to $f^{-1}(U)$ is flat.
    \item \textbf{(Dimension of fibers)} If $f : X \to Y$ is flat at $x$, then
        \[
            \dim \mathcal{O}_{X, x} = \dim \mathcal{O}_{Y, f(x)} + \dim
            f^{-1}(f(x)).
        \]
    \item \textbf{(Stalks pushforward)} If $f : X \to Y$ is flat, then
        $f(\mathcal{O}_{X, x}) = \mathcal{O}_{Y, f(x)}$.

    \item \textbf{(Topological quotient)} If $f : X \to Y$ is faithfully flat,
        then the topology on $Y$ is the quotient topology of $X$ induced by $f$.
\end{enumerate}

\begin{example}
    We now give a collection of examples of morphisms which are not flat. The
    last two are taken from \cite{MorphismNotes}.
    \begin{enumerate}
        \item Every blowup is a nonexample because the fibers are not
            equidimensional.

        \item Open immersions are flat because they induce isomorphisms on
            stalks.

        \item Closed immersions are never flat unless they are also open
            immersions (so of the form $X \to X \amalg Y$, the inclusion of a
            component).

        \item Coverings of the form $\Spec(R^n) \to \Spec(R)$ because free
            modules are injective so $\Tor^R_i(R^n, R) = 0$ for $i > 0$.

        \item Either coordinate projection $\mathbb{A}^2 \to \mathbb{A}$ is
            faithfully flat as $A[x_1, \dots, x_n]$ is flat over $A$.

        \item The normalization $\Spec(k[t^2, t^3]) \to \mathbb{A}^1$ is not
            flat, although it is surjective.
    \end{enumerate}
\end{example}

\subsection{Unramified morphisms}

\begin{definition}
    A morphism of local rings $f : (A, \mathfrak{m}_A) \to (B, \mathfrak{m}_B)$
    is \textit{unramified} if the induced extension of fields is separable.
\end{definition}

\begin{definition}
    A morphism of schemes $f : X \to Y$ is \textit{unramified} if the induced
    map on stalks is unramified as a morphism of local rings.
\end{definition}



\subsection{\'Etale morphisms}

The underlying problem with the Zariski topology is that there are simply not
enough open sets. For instance, every two paths $p_1, p_2 : [0, 1] \to
\mathbb{P}^1_{\mathbb{C}}$ are homotopic, but certainly the same is not true for
$\mathbb{P}_\mathbb{C}^1(\mathbb{C})$. To resolve this, we require a notion of
\textit{\'etale open sets}, which are essentially \'etale morphisms.

\'Etale morphisms are to the algebraic category as submersions are to the
differentiable one.

\begin{definition}
    Let $f : X \to Y$ be a morphism of schemes. We say $f$ is \text{\'etale} if
    it satifies one of two equivalent properties:
    \begin{enumerate}[(i)]
        \item \textbf{(Geometric definition)} $f$ is smooth of relative
            dimension zero.

        \item \textbf{(Algebraic definition)} $f$ is flat and unramified.
    \end{enumerate}
\end{definition}

There are many more definitions. For instance, one can define a class of
\textit{formally \'etale} morphisms via deformations and say an \'etale map is a
formally \'etale map subject to finiteness conditions. For varieties over a
closed field, one may say $f : V \to W$ is \'etale if it induces isomorphisms on
the Zariski tangent space of each point.

It is nonobvious that the given definitions are equivalent. The following
proposition proves it.

\begin{proposition}
    Let $f : X \to Y$ be a morphism of schemes. Then, the following are
    equivalent.
    \begin{enumerate}[(i)]
        \item $f$ is smooth of relative dimension zero.
        \item $f$ is flat and unramified.
    \end{enumerate}
\end{proposition}

The easiest case to classify are \'etale morphisms to a field.

\begin{theorem}
    Suppose $f : \Spec(S) \to \Spec(k)$ is \'etale. Then, $R \cong k_1 \times
    \cdots \times k_n$ where each $k_i$ is a separable finite field extensions
    of $k$.
\end{theorem}

\section{Topologies on $\Sch$}

We simply give the definitions and provide a few propositions that guide the
reader through the idea of the relation of the topology to the others in the
section.

Here is the general picture. The arrows correspond to increasing coarseness.
\[
    \begin{tikzcd}
        \text{canonical} \ar[r] &
        \text{fpqc} \ar[r] &
        \text{fppf} \ar[r] &
        \text{\'etale} \ar[r] &
        \text{Zariski}
    \end{tikzcd}
\]

The canoncial topology is the finest topology for which the functor of points
for a scheme $h_X$ is a sheaf. A topology is deemed \textit{subcanonical} if it
is coarser than the canoncial topology.

\subsection{The Zariski topology}

This is the most well-known, and it is strictly a formalization of the standard
point-set topoloy on a scheme. We define it anyways.

\begin{definition}
    A \textit{Zariski covering} of a scheme $U$ is a family of morphisms of
    schemes with fixed target $\{\phi_i : U_i \to U\}$ such that
    \begin{enumerate}
        \item Each $\phi_i$ is an open immersion.
        \item The family set-theoretically covers $U$, i.e. $\bigcup \phi_i(U_i)
            = U$.
    \end{enumerate}
\end{definition}

The Zariski topology is the most coarse topology we will define.

\begin{lemma}
    For any Zariski covering $\{\phi_i : U_i \to U\}$ for $U$ affine, there
    exists a Zariski covering $\{\psi_j : T_j \to U\}$ which is finite and
    refines the family $\{\phi_i\}$. Furthermore, each $T_j$ is a standard open
    of $U$.
\end{lemma}

\begin{proof}
    Affine schemes are quasicompact and the standard opens are a base for the
    topology.
\end{proof}

\begin{proof}
    The functor $(\Aff/S)_{\Zar} \to (\Sch/S)_{\Zar}$ for a scheme $S$ induces an
    equivalence of topoi $\Sh((\Aff/S)_{\Zar})$ and $\Sh((\Sch/S)_{\Zar})$.
\end{proof}

\begin{proof}
    Omitted. See \cite[\href{https://stacks.math.columbia.edu/tag/020W}{Tag
        020W}]{stacks-project}
\end{proof}

\subsection{The fpqc topology}

\begin{definition}
    A \textit{fpqc} covering is a family $\{\phi_i : U_i \to U\}_{i \in I}$ of
    morphisms of schemes of fixed target which satifies:
    \begin{enumerate}
        \item Each $\phi_i$ is flat.
        \item For each affine open $T \subset U$, there exists a $n > 1$ and a
            map $a : [n] \to I$ and affine opens $V_j \subset U_{a(j)}$ so that
            \[
                \bigcup_{j=1}^n \phi_{a(j)}(V_j) = U.
            \]
    \end{enumerate}
    In particular, this implies the usual covering condition that $\bigcup_{i
        \in I} \phi_i(U_i) = U$.
\end{definition}

\subsection{The \'etale topology}

\begin{definition}
    An \textit{\'etale covering} of a scheme $U$ is a family of morphisms of
    schemes with fixed target $\{\phi_i : U_i \to U \}$ such that

    \begin{enumerate}
        \item Each $\phi_i$ is etale.
        \item The family set-theoretically covers $U$, i.e. $\bigcup
            \phi_i(U_i) = U$.
    \end{enumerate}
\end{definition}

\begin{proposition}
    For a scheme $X$, the collection of families of \'etale coverings endow the
    following categories the structure of a site:

    \begin{enumerate}
        \item $X_\text{\'et}$.
        \item $(\Sch/X)_\text{\'et}$.
    \end{enumerate}
\end{proposition}

\begin{proof}
    Recall that $X_\text{\'et}$ (the \textit{small \'etale site}) is the
    category whose objects are \'etale morphisms to $X$. It is a fact that any
    triangle of schemes whose structure morphisms are \'etale is automatically
    an \'etale triangle (see \cite{stacks-project} for a proof). We check the
    three conditions for a Grothendieck pretology:
    \begin{enumerate}
        \item Isomorphisms are indeed \'etale.
        \item Compositions of \'etale morphisms are \'etale.
        \item The base change of an \'etale morphism is \'etale.
    \end{enumerate}

    The proof for the big \'etale site over $X$ is exactly the same.
\end{proof}

The following lemma implies the \'etale topology is subcanonical.

\begin{lemma}
    \'Etale coverings are fpqc.
\end{lemma}

\begin{proof}
    \'Etale morphisms are flat, and an \'etale covering is surjective on its
    target, so the first condition is satisfies.

    To check the quasicompactness property, let $V \subset U$ be an affine open,
    and let $V_{ij}$ be a covering of $\phi_i^{-1}(V)$ for some affine opens
    $V_{ij} \subset U_i$. \'Etale morphisms are open, so
    \[
        \bigcup_{i \in I} \bigcup_{j \in J} \phi(V_{ij})
    \]
    is an open covering of $V$. By the qc assumption on $V$, this covering has a
    finite refinement.
\end{proof}

\subsection{The fppf topology}

\begin{definition}
    A \textit{fppf covering} of a scheme $U$ is a family of morphisms of schemes
    with fixed target $\{\phi_i : U_i \to U\}$ such that
    \begin{enumerate}
        \item Each $\phi_i$ is flat.
        \item Each $\phi_i$ is locally of finite presentation.
        \item The family set theoretically covers $U$, i.e. $\bigcup \phi_i(U_i)
            = U$.
    \end{enumerate}
\end{definition}

\begin{proposition}
    The fppf coverings satisfy the definition of a pretopology.
\end{proposition}

\begin{proposition}
    Any Zariski or \'Etale coveing is an fppf covering.
\end{proposition}

\begin{proof}
    Clear from the definitions (open immersions are flat and locally of finite
    presentation because they are local isomorphisms, \'etale morphisms are flat
    and locally of finite presentation by definition).
\end{proof}

\section{Cohomology}

\subsection{Derived functors}

\begin{definition}
    Let $\mathcal{F} : \mathbf{A} \to \mathbf{B}$ be a functor. Suppose that
    \begin{enumerate}[(i)]
        \item $\mathcal{A}$ and $\mathbf{B}$ are abelian.
        \item For every $X \in \mathbf{A}$, there is an injective object $I$ and
            a monomorphism $X \to I$.
        \item $\mathcal{F}$ is left exact.
    \end{enumerate}
    Then, the \textit{$i$th derived functor} of $\mathcal{F}$, written $R^i
    \mathcal{F}$, is the $i$th cohomology group of any complex of the form
    \[
        0 \to I^1 \to I^2 \to \cdots
    \]
    where $X \to I^{\bullet}$ is an injective resolution of $X$.
\end{definition}

\begin{proposition}
    The definition of the functors $R^i \mathcal{F}$ is independent of the
    choice of injective resolution.
\end{proposition}

\begin{lemma}
    Given a short exact sequence $0 \to A \to B \to C \to 0$, there is a long
    exact sequence
    \[
        \begin{tikzcd}
            0 \ar[r] &
            F(A) \ar[r] &
            F(B) \ar[r] &
            F(C) \ar[dll] \\
            &
            R^1F(A) \ar[r] &
            R^1F(B) \ar[r] &
            R^1F(C) \ar[dll] \\
            &
            R^2F(A) \ar[r] &
            R^2F(B) \ar[r] &
            R^2F(C) \ar[r] & \cdots
        \end{tikzcd}
    \]
\end{lemma}

\begin{proof}
    Take an injective resolution of $A, B, C$ and apply the snake lemma.
\end{proof}

\subsection{Sheaf cohomology}

Fix a site $\mathcal{C}$ over the category $\Sch$ and consider the category of
abelian sheaves on $\mathcal{C}$. For $\mathcal{F} \in \Ab(\mathcal{C})$, we
define the \textit{$i$th sheaf cohomology of $\mathcal{F}$}, $H^i({-},
\mathcal{F})$ to be the $i$th right derived functor of $\Gamma({-},
\mathcal{F})$. Recall that $\Gamma({-}, \mathcal{F})$ is the global sections
functor for $\mathcal{F}$ on a covering $U$.

For this definition to make any sense, we require the following theorem.

\begin{theorem}
    The category of abelian sheaves on a site is abelian, and has enough
    injectives.
\end{theorem}

\begin{proof}
    See \cite[\href{https://stacks.math.columbia.edu/tag/03NT}{Tag
        03NT}]{stacks-project}.
\end{proof}

\subsection{\v{C}ech cohomology}

Unfortunately, (derived functor) sheaf cohomology is not easy to compute in
general. In this section, we introduce an alternative cohomology theory, which
is (sometimes) more computable and (sometimes) agrees with sheaf cohomology. All
material in this section follows the exposition in
\cite[\href{https://stacks.math.columbia.edu/tag/03OK}{Tag
    03OK}]{stacks-project}

\begin{definition}
    Let $\mathbf{C}$ be a category, $\mathcal{U} = \{U_i \to U\}_{i \in I}$ a
    family of morphisms with fixed target, and $\mathcal{F}$ an abelian
    presheaf. The \textit{\v{C}ech complex $\check{C}(\mathcal{U},
        \mathcal{F})$} by
    \[
        \prod_{i_0 \in I} \mathcal{F}(U_{i_0})
        \to \prod_{i_0, i_1\in I} \mathcal{F}(U_{i_0} \times_U U_{i_1})
        \to \prod_{i_0, i_1, i_2 \in I} \mathcal{F}(U_{i_0} \times_U U_{i_1}
        \times_U U_{i_2}) \to \cdots
    \]
    where the first term is in degree $0$. We allow the case $i_0 = i_1$, etc.

    The \textit{\v{C}ech cohomology groups} are defined by
    \[
        \check{H}^p(\mathcal{U}, \mathcal{F}) := H^p(\check{C}(\mathcal{U},
        \mathcal{F})).
    \]
\end{definition}

After some work, it is possible to show the following theorem.

\begin{theorem}
    The functors $\check{H}^p(\mathcal{U}, {-})$ for $p > 1$ are the right
    derived functors $\check{H}^0(\mathcal{U}, {-})$.
\end{theorem}

\begin{proof}[Outline]
    Demonstrate the \v{C}ech cohomology is a $\delta$-functor. Argue that
    $\check{H}^p(\mathcal{U}, \mathcal{I}) = 0$ for all $p > 0$ and injective
    $\mathcal{I}$. Invoke Grothendieck's result that effaceable
    $\delta$-functors are unique up to unique isomorphism, and that the derived
    functor is effeceable.
\end{proof}

We recall that there is an adjunction between the categories $\Ab(\mathcal{C})$
and $\PAb(\mathcal{C})$ given by sheafification on the left and the forgetful
functor on the right. This, purely formally, gives the following lemma.

\begin{lemma}
    The forgetful functor $\Ab(\mathcal{C}) \to \PAb(\mathcal{C})$ transforms
    injectives to injectives.
\end{lemma}

\begin{proof}
    As do all functors which possess a left adjoint.
\end{proof}

\begin{theorem}
    Let $\mathcal{C}$ be a site. For any covering $\mathcal{U} = \{U_i \to
    U\}_{i \in I}$ and any abelian sheaf $\mathcal{F}$ on $\mathcal{C}$ there
    exists a spectral sequence
    \[
        E_2^{p, q} = \check{H}^p(\mathcal{U}, \underline{H}^q(\mathcal{F}))
        \Rightarrow H^{p+q}(U, \mathcal{F}),
    \]
    where $\underline{H}^q(\mathcal{F})$ is the abelian presheaf $v \mapsto
    H^q(V, \mathcal{F})$.
\end{theorem}

\begin{proof}
    See \cite[\href{https://stacks.math.columbia.edu/tag/03OW}{Tag
        03OW}]{stacks-project}.
\end{proof}

\begin{remark}
    This spectral sequence is the Grothendieck spectral sequence corresponding
    to the composition
    \[
        \begin{tikzcd}
            \Ab(\mathcal{C}) \ar[r] &
            \PAb(\mathcal{C}) \ar[r, "\check{H}^0"] &
            \Ab.
        \end{tikzcd}
    \]
\end{remark}

\subsection{Cohomology of quasicoherent sheaves}

For better or for worse, the topologies we have introduced thus far cannot be
distinguished by the cohomology of $\mathcal{F}$ quasicoherent on $X$ a scheme.

As is usual by now, we follow quite closely the exposition in the Stacks
project. The following lemma allows us to reduce the analysis of a covering to a
single standard affine cover.

\begin{lemma}
    Let $\tau \in \{\text{fppf}, \text{\'etale}, \text{Zariski}\}$. Let $S$ be a
    scheme and $\mathcal{F}$ an abelian sheaf on $(\Sch/S)_\tau$, and let
    $\mathcal{U} = \{ U_i \to U\}$ be a standard $\tau$-covering of this site.
    Let $V = \coprod_{i \in I} U_i$. Then,
    \begin{enumerate}
        \item $V$ is affine.
        \item $\mathcal{V} = \{V \to U\}$ is a $\tau$-covering and an fpqc
            covering.
        \item The \v{C}ech complexes $\check{C}^{\bullet}(\mathcal{U},
            \mathcal{F})$ and $\check{C}^{\bullet}(\mathcal{V}, \mathcal{F})$
            agree.
    \end{enumerate}
\end{lemma}

\begin{lemma}
    Let $\mathcal{C}$ a site, $\mathcal{F}$ an abelian sheaf, $U$ an object of
    $\mathcal{C}$, $p \in \mathbb{Z}_{> 0}$. For all $\xi \in H^q(U,
    \mathcal{F})$, there exists a covering $\mathcal{U} = \{U_i \to U\}$ of
    $U$ so that $\restr{\xi}{U_i} = 0$ for all $i$.
\end{lemma}

We need one final lemma taken from faithfully flat descent.

\begin{lemma}
    If $A \to B$ is faithfully flat and $M$ is an $A$-module, then the complex
    $(B/A)_{\bullet} \otimes_A M$ is exact in positive degrees, and
    $H^0((B/A)_{\bullet} \otimes_A M) = M$.
\end{lemma}

Now, the main theorem. We present the proof in a way slightly reversed from the
Stacks project, in order to isolate the technical aspects for an inexperienced
reader (and, indeed, author). In particular, we have split the proof into a
lemma and a theorem.

\begin{lemma}
    Let $S$ be a scheme, $\mathcal{F}$ a quasicoherent $\mathcal{O}$-module. Let
    $\mathcal{C}$ be the big $\tau$-site over $S$, $(\Sch/S)_\tau$ for $\tau \in
    \{\text{fppf}, \text{\'etale}, \text{Zariski}\}$. For an object $f : U \to
    S$ in $\mathcal{C}$, $U$ affine, we have that
    \[
        H^p_{\tau}(U, \mathcal{F}^a) = 0,
    \]
    for $p > 0$.
\end{lemma}

\begin{proof}
    Begin with the case $p=1$. Choose a class $\xi \in H^1_\tau(U,
    \mathcal{F}^a)$. By Lemma $6.11$, there exists a covering $\mathcal{U} =
    \{U_i \to U\}$ so that $\restr{\xi}{U_i} = 0$. We may refine the cover so
    that it is a standard $\tau$-covering.

    The spectral sequence from Theorem $6.8$ shows that $\xi \in
    \check{H}^1(\mathcal{U}, \mathcal{F}^a)$. Lemma $6.10$ implies that, for the
    cover $\mathcal{V} = \{\coprod_{i \in I} U_i \to U\}$,
    $\check{H}^1(\mathcal{U}, \mathcal{F}^a) = \check{H}^1(\mathcal{V},
    \mathcal{F}^a)$. But, the cover $\mathcal{V}$ is of the form $\{\Spec(B) \to
    \Spec(A)\}$, so $f^{\ast}\mathcal{F} = \widetilde{M}$ for an $A$-module $M$.
    The \v{C}ech complex $\check{C}^\bullet(\mathcal{V}, \mathcal{F})$ is
    precisely $(B/A)_{\bullet} \otimes_A M$, however Lemma $6.11$ implies this
    complex has no higher cohomology.

    Now, the case $p>1$. We proceed essentially analogously. Choose $\xi \in
    H^p_\tau(U, \mathcal{F}^a)$. We may choose a standard $\tau$-covering
    $\mathcal{U}$ for which $\xi$ locally vanishes. The intersections $U_{i_0}
    \times_U \cdots \times_U U_{i_p}$ are affine, so by induction,
    $\check{H}^p(\mathcal{U}, \underline{H}^q(\mathcal{F}^a)) = 0$ for $0 < q <
    p$. This implies $\xi$ is the image of a $\check{\xi} \in
    \check{H}^p(\mathcal{U}, \mathcal{F}^a)$. Finally, replace $\mathcal{U}$
    with $\mathcal{V}$ as in the $p=1$ case, reduce to the complex of
    $A$-modules, and conclude that it has zero higher cohomology. Thus,
    $\check{\xi} = 0$ and moreover $\xi = 0$.
\end{proof}

\begin{theorem}
    Let $S$ be a scheme, $\mathcal{F}$ a quasicoherent $\mathcal{O}$-module. Let
    $\mathcal{C}$ be the big $\tau$-site over $S$, $(\Sch/S)_\tau$ for $\tau \in
    \{\text{fppf}, \text{\'etale}, \text{Zariski}\}$. Then,
    \[
        H^p(S, \mathcal{F}) = H^p_{\tau}(S, \mathcal{F}^a),
    \]
    for all $p \geq 0$.
\end{theorem}

\begin{proof}
    Let $f : U \to S$ be an element of the site $\mathcal{C}$. We need to show
    that $H^p(U, f^{\ast}\mathcal{F}) = H_{\tau}^p(U, \mathcal{F}^a)$. The case
    $p=0$ is a definition. For the case $p>1$, we invoke the previous lemma.

    Suppose $U$ is separated. We may choose an affine covering of $U$. This
    induces a fpqc covering $\mathcal{U}$. The intersections $U_{i_0} \times_U
    \cdots \times_U U_{i_p}$ are all affine as $U$ is separated. Thus, the rows
    of the spectral sequence in Theorem $6.8$ are all identically zero by the
    previous lemma, except the zeroeth row. So, the theorem implies
    \[
        H^p_\tau(U, \mathcal{F}^a) = \check{H}^p(\mathcal{U}, \mathcal{F}^a)
        = \check{H}^p(\mathcal{U}, \mathcal{F}),
    \]
    and the last group is famously equal to $H^p(U, \mathcal{F})$. There is some
    more work to extend to the nonseparated case, which we will not cover.
\end{proof}

\subsection{Analytic comparison theorem}

We simply state this amazing result.

\begin{theorem}
    Let $X$ be a nonsingular variety over $\mathbb{C}$. For any finite abelian
    group $\Lambda$ and $r \geq 0$, we have
    \[
        H^{r}(X_{\text{\'et}}, \Lambda) \cong H^r(X(\mathbb{C}), \Lambda),
    \]
    where we write $\Lambda$ for the constant sheaf with value $\Lambda$.
\end{theorem}

\section{Set theory}

This section is just a collection of interesting facts about the material
previously presented.

\subsection{A fpqc presheaf which may not be sheafified}

Unfortunately, with all the power that the general theory of sites and topoi
allow, some (allegedly) simple properties that one would expect a presheaf to
enjoy are not possible. We give an example of an fpqc sheaf for which there is
no sheafification. This example is taken from \cite{fpqc-sheaf}.

\begin{example}
    Fix a commutative ring $R$. We work in the relative category $\Ring/R$. For
    a prime ideal $\mathfrak{p} \in \Spec(A)$ for an $R$-algebra $A$, let
    $\kappa(\mathfrak{p})$ denote the fraction field of $A/\mathfrak{p}$. We
    define the following functor:
    \[
        \mathcal{F}(A) = \bigg\{ f : \Spec(A) \to \nu \mid \text{$f$ is locally
                constant and $\nu < \abs{\kappa(\mathfrak{p})}$ for all
                $\mathfrak{p} \in \Spec(A)$.} \bigg\}.
    \]
    Actually, we only know it is a map of underlying categorys $\Ring/R \to
    \Set$. However, if $f : A \to B$ is morphism of
    $R$-algebras, then $f^{-1}(\mathfrak{q})$ is a prime of $A$ which satisfies
    $\abs{\kappa(f^{-1}(\mathfrak{q}))} \leq \abs{\kappa(\mathfrak{q})}$, so the
    mapping is functorial.

    Let $\eta$ be any cardinal and let
    \[
        \mathcal{L}_{\eta}(A) = \bigg\{ f : \Spec(A) \to \eta \mid \text{$f$ is
            locally constant.} \bigg\}.
    \]
    Indeed, we immediately verify this is the sheaf $h_X = \Hom({-}, X)$ where
    $X = \coprod_{i \in \eta} \Spec(R)$. Consider a map which sends elements of
    a cardinal $\mu$ less than $\eta$ to themselves, otherwise the elements are
    mapped to zero. This gives a natural transformation $\mathcal{F} \to
    \mathcal{L}_{\eta}$. Thus, $\mathcal{F}$ could be sheafified to
    $\mathcal{G}$, then we would have a diagram of functors
    \[
        \begin{tikzcd}
            & \mathcal{G} \ar[dr] \\
            \mathcal{F} \ar[ur] \ar[rr] & & \mathcal{L}_{\eta}
        \end{tikzcd}
    \]
    We now note that given \textit{fixed} $A \in \Ring/R$, there exists a
    cardinal $\lambda$ for which $\abs{\kappa(\mathfrak{p})} < \lambda$ for all
    $\mathfrak{p} \in A$. Thus, the induced morphism $\mathcal{F}(A) \to
    \mathcal{L}_{\lambda}(A)$ is injective. But, and this is the crucial
    point, this implies $\mathcal{F}(A) \to \mathcal{G}(A)$ must be injective
    \textit{for every $A \in \Ring/R$!} From this we will derive a
    contradiction.

    Let $K/k$ be an extension of fields, so $R=k$. This is an fpqc covering of
    $k$. By the sheaf property, we have that the equalizer of
    \[
        \begin{tikzcd}
            \mathcal{F}(K) \ar[r, shift left=0.75ex] \ar[r, shift right=0.75ex]
            & \mathcal{F}(K \otimes_k K)
        \end{tikzcd}
    \]
    must inject into $\mathcal{G}(k)$. But, $\Spec(K)$ is a point, so the two
    arrows in the equalizer are the same, and $\mathcal{F}(K)$ is the equalizer.
    We have an injection
    \[
        \mathcal{F}(K) \xhookrightarrow{\quad} \mathcal{G}(k),
    \]
    but each object has the same cardinality as its argument field. We may
    choose an extension $K/k$ of arbitrary size, which is absurd.
\end{example}

\begin{remark}
    This example fails if we even move to the fppf topology, as $K/k$ is not an
    fppf covering for arbitrarily large extensions $K$.
\end{remark}

\subsection{Grothendieck universes}

\begin{definition}
    A \textit{Grothendieck universe} is a set $V$ so that
    \begin{enumerate}
        \item If $x \in V$ and $y \in x$, then $y \in x$.
        \item If $x, y \in V$, then $\{x, y\} \in V$.
        \item If $x \in U$, then $2^x \in U$.
        \item If $I \in V$ and $\{x_i\}_{i \in I}$ is a collection of elements
            of $V$, then $\bigcup_{i \in I} x_i \in V$.
    \end{enumerate}
\end{definition}

\begin{definition}
    A cardinal $\kappa$ is \textit{inaccessible} if there is no sequence of
    cardinal operations on smaller cardinals which reach $\kappa$. It is
    \textit{strongly inaccessible} if $\alpha < \kappa$ implies $2^{\alpha} <
    \kappa$ and any sum of less than $\kappa$ cardinals below $\kappa$ is less
    than $\kappa$.
\end{definition}

We have the following (amazing) fact.

\begin{theorem}
    The following statements are equivalent.
    \begin{enumerate}[(i)]
        \item For all $x$ a set, there exists a universe $V$ so that $x \in V$.
        \item For each cardinal $\kappa$, there is a strongly inaccessible
            cardinal greater than $\kappa$.
    \end{enumerate}
\end{theorem}

\begin{proposition}
    Let $\kappa$ be a strongly inaccessible cardinal. Then, $V_{\kappa} = \{x
    \text{ a set} \mid \abs{x} < \kappa\}$ is a Grothendieck universe.
\end{proposition}

\begin{proposition}
    Let $\kappa$ be a strongly inaccessible cardinal. Fix the universe
    $V_{\kappa}$. In the definition of the functor $\mathcal{F}$, restrict the
    codomain of locally constant functions to be cardinals less than $\kappa$.
    Then, $\mathcal{F}$ may be sheafified, and its sheaf is
    $\mathcal{L}_{\kappa}$.
\end{proposition}

\begin{proof}
    This is straightfoward from the definitions.
\end{proof}

\subsection{Lawvere-Tierney topologies}

This section is a stub---it will later be developed to address Lawvere-Tierney
topologies on a Grothendieck topos.

Classically, we may identify the set of subobjects of an object of
$\mathbf{Set}$ as the collection of monomorphisms $\{\phi : Z \to X\}$ modulo
isomorphism. But, this is in some sense, a second-order classification of
subobjects. We would like to be able to classify them in the first-order
language of the category itself (without having to invoke the notion of an
equivalence relation). In $\mathbf{Set}$, for each subobject $Z \subset X$,
there is a map $\chi_Z : X \to \{0, 1\}$ which is zero outside of $Z$ and one
inside of it. Thus, we may identify the subobjects of $X$ with the set $\Hom(X,
\{0, 1\})$. We recover the classical notion of a subobject in $\mathbf{Set}$ by
taking the following fibre product:

\begin{center}
    \begin{tikzcd}
        Z \ar[d, dashed] \ar[r, dashed] & \{1\} \ar[d, "1"] \\
        X \ar[r, "\chi_Z"] & \{0, 1\}
    \end{tikzcd}
\end{center}

We generalize this in the following way.

\begin{definition}
    Given a category $\mathcal{C}$ with finite limits, a \textit{subobject
        classifier} is an object $\Omega$ with a distinguished monomorphism
    $\text{true} : \mathbf{1} \to \Omega$, such that for every monomorphism $f :
    Z \to X$, there exists a unique $\chi_Z : X \to \Omega$ which allows $f$ to
    arise as the following fibre product:
    \begin{center}
        \begin{tikzcd}
            Z \ar[d, dashed, "f"] \ar[r, dashed] & \mathbf{1} \ar[d, "\text{true}"] \\
            X \ar[r, "\chi_Z"] & \Omega
        \end{tikzcd}
    \end{center}
\end{definition}

This definition is justified in $\mathbf{Set}$ by the previous discussion. For
more generate $\mathcal{C}$, we have the following proposition.

\begin{proposition}
    Suppose that $\mathcal{C}$ has finite limits and is locally small. Then,
    $\mathcal{C}$ has a subobject classifier if and only if the functor
    \begin{align*}
        \Sub : \mathcal{C} &\to \mathbf{Set} \\
        X &\mapsto \{Z \xhookrightarrow{} Z \},
    \end{align*}
    is representable. In this instance, $\Sub(-)$ is represented by the
    subobject classifier.
\end{proposition}

\begin{proof}
    By definition, the existence of a subobject classifier gives a bijection
    \[
        \theta_X : \Sub(X) \cong \Hom(X, \Omega)
    \]
    defined by sending a subobject $Z$ to the characteristic function $\chi_Z$
    induced by pullback. To show that this is indeed natural, consider a
    morphism $f : X \to Y$ and subobjects $Z \subset X$, $W \subset Y$. The
    functor $\Sub(-)$ is contravariant in pullbacks, so by considering the
    following diagram,
    \begin{center}
        \begin{tikzcd}
            Z \ar[d, hook] \ar[r] &
            W \ar[d, hook] \ar[r] &
            \mathbf{1} \ar[d, "\text{true}"] \\


            X \ar[r, "f"] &
            Y \ar[r] &
            \Omega
        \end{tikzcd}
    \end{center}
    we can observe (by the pasting law for pullbacks) that $\theta_X$ is indeed
    natural.

    Now, suppose we are given such a $\theta_X$ for all $X \in \mathcal{C}$,
    natural in $X$ with representing object $\Theta$. In particular, we have
    that $\Sub(\Omega) \cong \Hom(\Omega, \Omega)$, so there is a unique
    subobject $\Omega_0$ corresponding to the identity on $\Omega$. Now,
    consider any map $f : X \to \Omega$. By naturality of $\theta_X$, we have a
    diagram
    \[
        \begin{tikzcd}
            \Sub(\Omega) \ar[r, "\theta_\Omega"] \ar[d, "\Sub(f)"] &
            \Hom(\Omega, \Omega) \ar[d, "h_\Omega(f)"] \\
            \Sub(X) \ar[r, "\theta_X"] &
            \Hom(X, \Omega)
        \end{tikzcd}
        \quad\quad
        \begin{tikzcd}
            \Omega_0 \ar[d, mapsto] \ar[r, mapsto] & 1 \ar[d, mapsto] \\
            S \ar[r, mapsto] & \phi
        \end{tikzcd}
    \]
    for some $\phi$. In particular, for any subobject $S \subset X$, we can
    write $S = \Sub(f)(\Omega_0)$, and indeed, recover it as a pullback of
    $\Omega_0$ along some uniquely specified $\phi$. This produces the diagram:
    \[
        \begin{tikzcd}
            S \ar[r, dashed] \ar[d, dashed] & \Omega_0 \ar[d] \\
            X \ar[r, "\phi"] & \Omega
        \end{tikzcd}
    \]
    Thus, if we can demonstrate that $\Omega_0$ is terminal in $\mathcal{C}$,
    then we are finished. In the above diagram, let $S = X$. Then, we achieve a
    map $f_1 : X \to \Omega_0$. Suppose there were another map $f_2 : X \to
    \Omega_0$. Then, the square equipped with that map in its top horizontal row
    would still be (trivially) a pullback diagram. If we label the monic
    $\Omega_0 \to \Omega$ by $t$, then both $t \circ f_1$ and $t \circ f_2$ must
    equal $\phi$ by the uniqueness of $\phi$. But, $t$ is monic, so $f_1 = f_2$.
    This construction gives a unique morphism $X \to \Omega_0$ for all $X \in
    \mathcal{C}$, so it is terminal. The morphism $t$ is the so-called
    ``$\text{true}$'' morphism.
\end{proof}

\begin{corollary}
    Suppose $\mathcal{C}$ has finite limits, is locally small, and enjoys a
    subobject classifier. Then, $\mathcal{C}$ is well-powered.
\end{corollary}



\bibliography{bibliography}
\bibliographystyle{ieeetr}

\end{document}
